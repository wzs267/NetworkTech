% !TeX program = xelatex
\documentclass[a4paper,12pt]{article}
\usepackage{xeCJK}
\usepackage{fontspec}
\usepackage{graphicx}
\usepackage{float}
\usepackage{geometry}
\usepackage{hyperref}
\usepackage{listings}
\geometry{left=2.5cm,right=2.5cm,top=2.5cm,bottom=2.5cm}
\setCJKmainfont{SimSun}
\setmainfont{Times New Roman}
\title{PPPoE 服务器的配置和应用}
\author{王泽舜 \quad 学号:2310655}
\date{\today}

\begin{document}
\maketitle

\section*{实验目的}
- 掌握 PPPoE 服务器的基本配置方法(认证协议、地址池、虚拟模板与物理接口)。
\section*{实验要求}
- PPPoE 服务器配置和应用实验在虚拟仿真环境下完成,要求如下:
	1. 仿真有线局域网接入互联网的场景,正确配置 PPPoE 服务器的认证协议、地址池、虚拟模板和物理接口,使内网用户经认证后才能正常访问外部互联网。
	2. 仿真家庭网络中,无线和有线终端(主机、智能电话等)连入小型路由器,由小型路由器统一接入互联网服务运营商 PPPoE 服务器的场景。对小型路由器和 PPPoE 服务器进行设置,使家庭网络中的用户经认证后才能正常访问外部互联网。

\section*{实验拓扑与环境}
实验拓扑如图所示,采用三台设备模拟:`server3`(AAA/RADIUS 服务器)、`router2`(ISP / PPPoE 服务端)和连接到外网的路由器(模拟外网,192.168.3.0 网段)。

\begin{figure}[htbp]
\centering
\includegraphics[width=0.85\textwidth]{pictures/网络拓扑.png}
\caption{实验网络拓扑}
\end{figure}

\section*{设备与 IP 规划}
- AAA 服务器(`server3`):192.168.2.3(作为 RADIUS 认证、账号数据库)。
- ISP 路由器(`router2`):对外接口连接外网,内侧用于 PPPoE 服务,地址池用于分配给 PPPoE 客户端。
- 内网段(示例):192.168.1.0/24,地址池范围 `192.168.1.100`–`192.168.1.200`。

\section*{配置要点与步骤}
以下按 `router2`(ISP)与 `server3`(AAA)两端配置说明:

\subsection*{1) AAA/RADIUS(server3)}
在 `server3` 上配置用户账号与 RADIUS 服务,供 PPPoE 验证使用。相关截图见图\ref{fig:aaaconf}。

\begin{figure}[H]
\centering
\includegraphics[width=0.8\textwidth]{pictures/AAAServer网络和账号配置.png}
\caption{AAA 服务器网络与账号配置示意}\label{fig:aaaconf}
\end{figure}

\subsection*{2) ISP 路由器(router2)配置指令}
在 `router2` 上启用 `aaa new-model`,配置 RADIUS 服务器、地址池、虚拟模板并将物理接口启用 PPPoE:

\begin{lstlisting}
aaa new-model
aaa authentication ppp myPPPoE group radius
radius-server host 192.168.2.3 auth-port 1645 key radius123
ip local pool myPool 192.168.1.100 192.168.1.200
interface virtual-template 1
 ip unnumber gig0/1
 peer default ip address pool myPool
 ppp authentication chap myPPPoE
exit
bba-group pppoe myBBAGroup
 virtual-template 1
exit
interface gig0/1
 pppoe enable group myBBAGroup
exit
\end{lstlisting}

说明:
- `radius-server host` 指向 `server3` 的地址与共享密钥(示例为 `radius123`)。
- `ip local pool myPool` 定义给拨号客户端分配的地址段。
- `virtual-template` 用作 PPPoE 会话模板,包含认证方式与地址分配策略;通过 `bba-group` 将模板绑定到物理接口的 PPPoE 功能上。

\section*{测试方法与结果}
测试流程:
1. 在内网主机端发起 PPPoE 拨号(作为 PPPoE 客户端)。
2. 路由器 `router2` 接收拨号请求并将认证请求转发到 `server3`(RADIUS)。
3. AAA 验证通过后,`router2` 从 `myPool` 分配 IP,建立隧道并允许访问外部网络。

测试结果总结:
- 客户端成功通过 CHAP(或 PAP/CHAP 任选)方式被 `server3` 验证。
- 客户端获得来自地址池 `192.168.1.100`–`192.168.1.200` 的 IP 地址,并可访问模拟外网(192.168.3.0/24)。
- 认证失败的客户端无法获得地址或访问外部网络,满足基于认证的接入控制要求。

(实验中相关截图证明了 AAA 配置与拨号成功,已插入上文图片。)
实验中获取的关键测试截图如下,分别展示拨号成功、地址分配以及内网访问外网与外网对内网的连通性验证。

\begin{figure}[htbp]
\centering
\includegraphics[width=0.48\textwidth]{pictures/PPPoE连接成功.png}
\includegraphics[width=0.48\textwidth]{pictures/为内网主机分配ip.png}
\caption{PPPoE 拨号成功与为内网主机分配 IP}
\end{figure}

\begin{figure}[htbp]
\centering
\includegraphics[width=0.48\textwidth]{pictures/内网访问外网web.png}
\includegraphics[width=0.48\textwidth]{pictures/外网ping内网.png}
\caption{内网访问外网的 Web 测试与外网对内网的 Ping 验证}
\end{figure}

\section*{问题与注意事项}
- RADIUS 服务器和路由器之间的时间同步与共享密钥必须一致,否则认证会失败。
- 地址池大小应根据并发拨号的最大数量合理设置,避免地址耗尽。
- 在真实生产环境中需注意安全(如使用更安全的 AAA 协议、限制管理访问等)。

\section*{结论}
通过本次实验,可以掌握 PPPoE 服务端的典型配置流程:启用 AAA 模型、配置 RADIUS 服务器、定义地址池与虚拟模板,并在物理接口上启用 PPPoE。实验验证了经认证的用户可获得地址并访问外网,未认证用户被阻断,符合预期实验要求。

\section*{编译说明}
建议使用 XeLaTeX 编译:
\begin{verbatim}
xelatex lab8_report.tex
\end{verbatim}

\end{document}
