\documentclass[12pt,a4paper]{article}
\usepackage[margin=1in]{geometry}
\usepackage{xeCJK}
\usepackage{graphicx}
\usepackage{booktabs}
\usepackage{array}
\usepackage{listings}
\usepackage{xcolor}
\usepackage{hyperref}
\usepackage{fancyhdr}

\setCJKmainfont{SimSun}
\setCJKmonofont{Courier New}

\lstset{
    language=C,
    basicstyle=\ttfamily\small,
    keywordstyle=\color{blue},
    commentstyle=\color{gray},
    stringstyle=\color{red},
    breaklines=true,
    showstringspaces=false,
    tabsize=4,
    frame=single
}

\pagestyle{fancy}
\fancyhf{}
\fancyhead[C]{网络技术与应用实验课 — 实验三报告}
\fancyfoot[C]{\thepage}

\title{\Large\textbf{通过编程获取 IP 地址与 MAC 地址的对应关系\\实验报告}}
\author{}
\date{2025年11月26日}

\begin{document}
\maketitle

\section*{实验基本信息}
\begin{center}
\begin{tabular}{|c|c|c|c|}
\hline
\textbf{实验名称} & 通过编程获取IP地址与MAC地址的对应关系 & \textbf{班级} & 0416 \\
\hline
\textbf{学生姓名} & 王泽舜 & \textbf{学号} & 2310655 \\
\hline
\textbf{指导老师} & 师建新 & \textbf{实验地点} & 实验楼A区312 \\
\hline
\textbf{实验时间} & 11.26 & \textbf{报告日期} & 2025年11月26日 \\
\hline
\end{tabular}
\end{center}

\section{作业说明}
通过编程获取IP地址与MAC地址的对应关系实验,要求如下:
\begin{enumerate}
  \item 在 IP 数据报捕获与分析编程实验的基础上,学习 WinPcap/NPcap 的数据包发送方法(\texttt{pcap\_sendpacket})。
  \item 通过 WinPcap/NPcap 编程,主动获取或被动捕获 IP 与 MAC 的映射关系(以 ARP 为主)。
  \item 程序要具有输入 IP 地址并显示对应 MAC 地址的界面;界面可以是命令行界面,需以简单明了的方式在屏幕上显示。 
  \item 程序结构清晰,具有较好的可读性。
\end{enumerate}

\section{设计与实现}
\subsection{总体思路}
本实验采用主动发送 ARP 请求并同时监听 ARP 回复的方法:
\begin{itemize}
  \item 使用 NPcap 提供的 \texttt{pcap\_open\_live} 打开用户选择的网络接口;
  \item 构造以太网帧(以太头 + ARP payload),通过 \texttt{pcap\_sendpacket} 发送 ARP request;
  \item 在同一接口上设置 BPF 过滤器 `arp`,捕获 ARP reply 并解析(opcode==2 且 sender IP 匹配目标 IP);
  \item 为提高可靠性,发送后调用 Windows API `SendARP` 作为后备,若内核 ARP 缓存已有条目可立即返回。 
\end{itemize}

\subsection{模块划分}
\begin{itemize}
  \item \textbf{arp\_sender}: 构造并发送 ARP 请求(`arp\_sender.c`/`arp\_sender.h`)。
  \item \textbf{arp\_parser}: 解析捕获到的以太网帧并提取 ARP 回复的 sender IP/MAC(`arp\_parser.c`/`.h`)。
  \item \textbf{main\\lab3}: 命令行界面、设备选择、发送/捕获协调、结果显示与清理(`main\_lab3.c`)。
  \item \textbf{工具/说明}: `README.md` 包含编译与运行说明。 
\end{itemize}

\subsection{关键代码片段}
构造并发送 ARP 请求(节选,详见 `arp\_sender.c`):
\begin{lstlisting}
// Ethernet header: dst=ff:ff:ff:ff:ff:ff, src=src\_mac, ethertype=0x0806
// ARP payload: htype=1, ptype=0x0800, hlen=6, plen=4, opcode=1 (request)
pcap\_sendpacket(handle, packet, sizeof(packet));
\end{lstlisting}

捕获并解析 ARP 回复(节选,详见 `arp\_parser.c`):
\begin{lstlisting}
// parse ethernet type, then parse arp payload
if (ethertype == 0x0806 && opcode == 2) {
    // extract sender\_hw and sender\_ip
}
\end{lstlisting}

主程序流程要点(节选,详见 `main\_lab3.c`):
\begin{enumerate}
  \item 列举并选择设备(`pcap\_findalldevs`);
  \item 读取本机接口 MAC 与 IPv4;
  \item 用户输入目标 IPv4;
  \item 发送 ARP 请求,设置 BPF 过滤器 `arp`;
  \item 监听并解析 ARP reply,或使用 `SendARP` 作为后备;
  \item 将 IP → MAC 映射显示在命令行,并在退出前等待用户确认。
\end{enumerate}

\section{实验结果}
\subsection{运行片段}
以下为一次实际运行的关键输出(用户交互节选):
\begin{verbatim}
Available devices:
1. \Device\NPF\_{...} (WAN Miniport (Network Monitor))
...
6. \Device\NPF\_{5864...} (Intel(R) Wi-Fi 6E AX211 160MHz)
...
Select device number: 6
Using source MAC: D4:D8:53:33:E7:6C
Enter IPv4 address to query (e.g. 192.168.1.10): 10.130.0.1
Using source IP: 10.130.60.179
ARP request sent, waiting for reply (2s)...
SendARP result: 10.130.0.1 => 00:00:5E:00:01:FE
Press Enter to exit...
\end{verbatim}

系统 `arp -a` 同一时间段返回:
\begin{verbatim}
  10.130.0.1            00-00-5e-00-01-fe     dynamic
\end{verbatim}

\subsection{分析与结论}
\begin{itemize}
  \item 程序通过 `pcap\_sendpacket` 主动发送 ARP 请求,并能在多数情形下通过 `pcap` 捕获到 ARP 回复;在本次环境中,Windows 内核可能已经预先解析或路由器响应导致 `SendARP` 能直接返回 MAC,其结果与系统 ARP 缓存一致,验证了实现正确性。
  \item 本次实验实现并验证了使用 NPcap 的数据包发送与捕获能力,能够主动获取本地网络内的 IP→MAC 映射,完成了实验要求。
\end{itemize}



\end{document}
